\documentclass[aps,prb,twocolumn,superscriptaddress,floatfix,longbibliography,10pt]{revtex4-2}

\usepackage[utf8]{inputenc}
\usepackage[spanish]{babel}
\usepackage{graphicx}
\usepackage{amsmath}
\usepackage{subcaption}
\usepackage{wrapfig} 
\usepackage[export]{adjustbox}

\usepackage{amsmath,amssymb} % math symbols
\usepackage{bm} % bold math font
\usepackage{graphicx} % for figures
\usepackage{comment} % allows block comments
\usepackage{textcomp} % This package is just to give the text quote '
%\usepackage{ulem} % allows strikeout text, e.g. \sout{text}

\usepackage[spanish]{babel}
% By dafault, spanish changes to a comma as decimal separator; to change to a dot, you can use \decimalpoint:
\decimalpoint

\usepackage{enumitem}
\setlist{noitemsep,leftmargin=*,topsep=0pt,parsep=0pt}

\usepackage{xcolor} % \textcolor{red}{text} will be red for notes
\definecolor{lightgray}{gray}{0.6}
\definecolor{medgray}{gray}{0.4}

%Para las tablas
\usepackage{multirow}

\usepackage{hyperref}
\hypersetup{
colorlinks=true,
urlcolor= blue,
citecolor=blue,
linkcolor= blue,
bookmarks=true,
bookmarksopen=false,
}

% Code to add paragraph numbers and titles
\newif\ifptitle
\newif\ifpnumber
\newcounter{para}
\newcommand\ptitle[1]{\par\refstepcounter{para}
{\ifpnumber{\noindent\textcolor{lightgray}{\textbf{\thepara}}\indent}\fi}
{\ifptitle{\textbf{[{#1}]}}\fi}}
% \ptitletrue  % comment this line to hide paragraph titles
% \pnumbertrue  % comment this line to hide paragraph numbers

% minimum font size for figures
\newcommand{\minfont}{6}

% Uncomment this line if you prefer your vectors to appear as bold letters.
% By default they will appear with arrows over them.
% \renewcommand{\vec}[1]{\bm{#1}}

%Cambiar Cuadros por Tablas y lista de...
%\renewcommand{\listtablename}{Índice de tablas}
\renewcommand{\tablename}{Tabla}
\renewcommand{\date}{Fecha}

%Para importar imágenes desde una carpeta:
\graphicspath{ {C:/Users/lupam/OneDrive/Escritorio/GitHub/Vapor_en_burbuja_inicial_python/Tesis_informe/Figures}}

\usepackage[bottom]{footmisc} %para que las notas al pie aparezcan en la misma página

\begin{comment}

%Comandos de interés:

* Para ordenar el documento:
\section{Introducción}
\section{\label{sec:Formatting}Formatting} %label para luego hacer referencia a esa sección

\ptitle{Start writing while you experiment} %pone nombre y título al documento dependiendo de si en el header están los comandos \ptitletrue y \pnumbertrue

* Ecuaciones:
\begin{equation}
a^2+b^2=c^2 \,.
\label{eqn:Pythagoras}
\end{equation}

* Conjunto de ecuaciones:
\begin{eqnarray}
\label{eqn:diagonal}
\nonumber d & = & \sqrt{a^2 + b^2 + c^2} \\
& = & \sqrt{3^2+4^2+12^2} = 13
\end{eqnarray}

* Para hacer items / enumerar:
\begin{enumerate}
  \item
\end{enumerate}

\begin{itemize}
  \item
\end{itemize}

* Figuras:
\begin{figure}[h]
    \includegraphics[clip=true,width=\columnwidth]{pixel-compare}
    \caption{}
     \label{fig:pixels}
\end{figure}

* Conjunto de figuras:
(no recuerdo)


* Para hacer referencias a fórmulas, tablas, secciones, ... dentro del documento:
\ref{tab:spacing}

* Para citar
Elementos de .bib
\cite{WhitesidesAdvMat2004}
url
\url{http://www.mendeley.com/}\\

* Agradecimientos:
\begin{acknowledgments}
We acknowledge advice from Jessie Zhang and Harry Pirie to produce Fig.\ \ref{fig:pixels}.
\end{acknowledgments}

* Apéndice:
\appendix
\section{\label{app:Mendeley}Mendeley}

* Bibliografía:
\bibliography{Hoffman-example-paper}

\end{comment}


\begin{document}

% Allows to rewrite the same title in the supplement
\newcommand{\mytitle}{\textcolor{red}{TESIS - trabajo realizado hasta 27 de Octubre de 2022}}

\title{\mytitle}

\author{Pablo Chehade \\
    \small \textit{pablo.chehade@ib.edu.ar} \\
    \small \textit{Instituto Balseiro, CNEA-UNCuyo, Bariloche, Argentina, 2022} \\}


\begin{abstract}
  \begin{itemize}
    \item Se estudió numéricamente la expansión de una burbuja de cavitación producida mediante un láser
    \item La evolución temporal se divide en dos partes, cada una de ellas gobernada por fenómenos físicos distintos.
    \item La primera consta desde la creación de la burbuja hasta la expansión rápida al radio máximo. El fenómeno que creemos ocurre es la interacción del láser con los electrones de burbujas microscópicas (bubston) para dar lugar a una avalancha de electrones impulsora de la expansión (fenómeno electromagnético y fluidodinámico). El fenómeno se debe principalmente a la acción de los electrones y las moléculas ionizadas que rodean exteriormente la burbuja.
    \item La segunda, desde el radio máximo hasta su implosión al radio mínimo. En este momento el efecto de los electrones se diluye y comienzan a preponderar efectos de conducción del calor, cambios en la masa contenida dentro de la esfera debido a reacciones químicas, difusión y condensación, entre otros. El fenómeno se debe principalmente a la acción de los iones y moléculas dentro y fuera de la burbuja (fenómeno fluidodinámico, termodinámico y químico).
    \item Se exploraron además distintos métodos numéricos para resolver este problema complejo que inherentemente es del tipo stiff debido a las diferencias de órdenes de magnitud entre las constantes de tiempo involucradas.
  \end{itemize}


\end{abstract}

\maketitle

\section*{Índice}
\textcolor{red}{Buscar en internet cómo hacer un índice en Latex}

\section*{Introducción}

\section*{Capítulo I:}

\section*{Capítulo II:}

\section*{Conclusiones}
\begin{enumerate}
  \item Se calculó numéricamente el radio máximo de la burbuja, obteniendo un valor 10 veces mayor a los resultados experimentales (y teóricos del Bunkin)
  \item Se calculó numéricamente el tiempo al que la burbuja alcanza el radio máximo, obteniendo un valor 10 veces mayor a los resultados experimentales (y teóricos del Bunkin)
  \item Ídem para la energía, salvo por un factor 600
  \item Se encontró que el problema es de naturaleza stiff, lo cual permitió elegir un mejor método numérico que el que se venía usando. Esto permitió disminuir enormemente el tiempo de cómputo
  \item \textcolor{red}{Mencionar algo del cluster}






\end{enumerate}




\bibliography{Chehade_tesis.bib}

\end{document}






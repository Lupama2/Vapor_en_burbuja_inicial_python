\documentclass[aps,prb,twocolumn,superscriptaddress,floatfix,longbibliography,10pt]{revtex4-2}

% \documentclass[aps,prb,superscriptaddress,floatfix,longbibliography,13pt]{revtex4-2}


\usepackage[utf8]{inputenc}
\usepackage[spanish]{babel}
\usepackage{graphicx}
\usepackage{amsmath}
\usepackage{subcaption}
\usepackage{wrapfig} 
\usepackage[export]{adjustbox}

\usepackage{amsmath,amssymb} % math symbols
\usepackage{bm} % bold math font
\usepackage{graphicx} % for figures
\usepackage{comment} % allows block comments
\usepackage{textcomp} % This package is just to give the text quote '
%\usepackage{ulem} % allows strikeout text, e.g. \sout{text}

\usepackage[spanish]{babel}
% By dafault, spanish changes to a comma as decimal separator; to change to a dot, you can use \decimalpoint:
\decimalpoint

\usepackage{enumitem}
\setlist{noitemsep,leftmargin=*,topsep=0pt,parsep=0pt}

\usepackage{xcolor} % \textcolor{red}{text} will be red for notes
\definecolor{lightgray}{gray}{0.6}
\definecolor{medgray}{gray}{0.4}

%Para las tablas
\usepackage{multirow}

\usepackage{hyperref}
\hypersetup{
colorlinks=true,
urlcolor= blue,
citecolor=blue,
linkcolor= blue,
bookmarks=true,
bookmarksopen=false,
}

% Code to add paragraph numbers and titles
\newif\ifptitle
\newif\ifpnumber
\newcounter{para}
\newcommand\ptitle[1]{\par\refstepcounter{para}
{\ifpnumber{\noindent\textcolor{lightgray}{\textbf{\thepara}}\indent}\fi}
{\ifptitle{\textbf{[{#1}]}}\fi}}
\ptitletrue  % comment this line to hide paragraph titles
\pnumbertrue  % comment this line to hide paragraph numbers

% minimum font size for figures
\newcommand{\minfont}{6}

% Uncomment this line if you prefer your vectors to appear as bold letters.
% By default they will appear with arrows over them.
% \renewcommand{\vec}[1]{\bm{#1}}

%Cambiar Cuadros por Tablas y lista de...
%\renewcommand{\listtablename}{Índice de tablas}
\renewcommand{\tablename}{Tabla}
\renewcommand{\date}{Fecha}

%Para importar imágenes desde una carpeta:
\graphicspath{ {C:/Users/lupam/OneDrive/Escritorio/GitHub/Vapor_en_burbuja_inicial_python/Tesis_informe/Figures}}

\usepackage[bottom]{footmisc} %para que las notas al pie aparezcan en la misma página

\begin{comment}

%Comandos de interés:

* Para ordenar el documento:
\section{Introducción}
\section{\label{sec:Formatting}Formatting} %label para luego hacer referencia a esa sección

\ptitle{Start writing while you experiment} %pone nombre y título al documento dependiendo de si en el header están los comandos \ptitletrue y \pnumbertrue

* Ecuaciones:
\begin{equation}
a^2+b^2=c^2 \,.
\label{eqn:Pythagoras}
\end{equation}

* Conjunto de ecuaciones:
\begin{eqnarray}
\label{eqn:diagonal}
\nonumber d & = & \sqrt{a^2 + b^2 + c^2} \\
& = & \sqrt{3^2+4^2+12^2} = 13
\end{eqnarray}

* Para hacer items / enumerar:
\begin{enumerate}
  \item
\end{enumerate}

\begin{itemize}
  \item
\end{itemize}

* Figuras:
\begin{figure}[h]
    \includegraphics[clip=true,width=\columnwidth]{pixel-compare}
    \caption{}
     \label{fig:pixels}
\end{figure}

* Conjunto de figuras:
(no recuerdo)


* Para hacer referencias a fórmulas, tablas, secciones, ... dentro del documento:
\ref{tab:spacing}

* Para citar
Elementos de .bib
\cite{WhitesidesAdvMat2004}
url
\url{http://www.mendeley.com/}\\

* Agradecimientos:
\begin{acknowledgments}
We acknowledge advice from Jessie Zhang and Harry Pirie to produce Fig.\ \ref{fig:pixels}.
\end{acknowledgments}

* Apéndice:
\appendix
\section{\label{app:Mendeley}Mendeley}

* Bibliografía:
\bibliography{Hoffman-example-paper}

\end{comment}


\begin{document}

% Allows to rewrite the same title in the supplement
\newcommand{\mytitle}{\textcolor{red}{Cálculos Computacionales en Plasmas Producidos por Cavitación Láser y colapso de burbujas - trabajo realizado hasta 27 de Octubre de 2022}}

\title{\mytitle}

\author{Pablo Chehade \\
    \small \textit{pablo.chehade@ib.edu.ar} \\
    \small \textit{Instituto Balseiro, CNEA-UNCuyo, Bariloche, Argentina, 2022} \\}


\begin{abstract}
  \begin{itemize}
    \item \textcolor{red}{TEST}
    \item Se estudió numéricamente la expansión de una burbuja de cavitación láser. \textcolor{red}{y la compresión}
    \item La evolución temporal se divide en dos partes, cada una de ellas gobernada por fenómenos físicos distintos.
    \item La primera consta desde la creación de la burbuja hasta la expansión rápida al radio máximo. El fenómeno que creemos ocurre es la interacción del láser con los electrones de burbujas microscópicas (bubston) para dar lugar a una avalancha de electrones impulsora de la expansión (fenómeno electromagnético y fluidodinámico). El fenómeno se debe principalmente a la acción de los electrones y las moléculas ionizadas que rodean exteriormente la burbuja.
    \item La segunda, desde el radio máximo hasta su implosión al radio mínimo. En este momento el efecto de los electrones se diluye y comienzan a preponderar efectos de conducción del calor, cambios en la masa contenida dentro de la esfera debido a reacciones químicas, difusión y condensación/evaporación, entre otros. El fenómeno se debe principalmente a la acción de los iones y moléculas dentro y fuera de la burbuja (fenómeno fluidodinámico, termodinámico y químico).
    \item Se exploraron además distintos métodos numéricos y técnicas de resolución computacionales para resolver este problema complejo que inherentemente es del tipo stiff debido a las diferencias de órdenes de magnitud entre las constantes de tiempo involucradas.
  \end{itemize}


\end{abstract}

\maketitle

\section*{Índice}
\textcolor{red}{Buscar en internet cómo hacer un índice en Latex}

\section*{Introducción}


















\section*{Capítulo I: Evolución hasta el radio máximo}

\ptitle{Resumen del capítulo. En este capítulo se discute el proceso a partir del cual un láser incidente en un medio líquido es capaz de producir una burbuja de cavitación}
\begin{itemize}
  \item El proceso se desarrolla en \textcolor{red}{Referencia al Bunkin}
  \item Se parte de un medio líquido, en nuestro caso agua deuterada, en el cual se asume existen micro burbujas de gas (bubstons) agrupadas en clusters en la zona focal del haz de luz. 
  \item Se asume que dentro de cada bubston hay inicialmente un electrón libre
  \item Bajo estas condiciones, se demuestra en el paper que bajo determinadas condiciones de la burbuja y de la intensidad del haz, se desarrolla una avalancha de electrones
  \item Esta produce una presión eléctrica que expande el bubston hasta que los bubstons del cluster se fusionan entre sí y forman una gran burbuja o "núcleo". 
  \item Este efecto se denomina SOC
  \item Explicar brevemente qué ocurre luego
  \item Las cuentas se hacen en el sistema CGS
  \item Contar que el proceso consta de 3 etapas

\end{itemize}

\ptitle{Situación física inicial}
\begin{itemize}
  \item Resumen: en esta sección se explica la situación física inicial antes de la incidencia del láser.
  \item Se parte de un medio líquido en el que se encuentran clusters de burbujas estables (bubstons) de radio $R_0$.
  \item Explicar cómo calcular R0 y dar el valor aproximada
  \item La estabilidad de los bubstons se basa en la condición $R_0 << l_{em}$. Explicar
  \item Se considera que dentro de la burbuja hay al menos un electrón libre
  \item Definir R0 en algún lado
\end{itemize}



\subsection{Avalancha de electrones dentro de los bubston}

En esta sección se explicará el proceso a través del cual aumenta la cantidad de electrones $N_e$ dentro de cada bubston al incidir un láser. Se parte de la hipótesis de que existe al menos un electrón libre en cada uno de ellos. Además, se asume ocurre rápidamente de modo que su radio se mantiene aproximadamente constante. Se encontrará que el aumento de $N_e$ no es indefinido, sino que se alcanza un valor máximo a partir del cual se mantiene constante. También se hallará una condición sobre la intensidad del haz de luz necesaria para que ocurra la avalancha.

\textcolor{red}{La suposición de que el radio se mantiene constante y el proceso ocurre rápidamente es necesaria para, creo, asegurar que solo ocurren 2 fenómenos que contribuyen a la avalancha}


\ptitle{La avalancha comienza con un electrón dentro de la burbuja inmerso en el campo electromagnético producido por el láser}

Como se mencionó anteriormente, el proceso comienza con al menos un electrón libre dentro de cada bubston. Al incidir el láser, el electrón interactúa con el campo electromagnético oscilatorio y se acelera. En su movimiento choca con las paredes de la burbuja con una frecuencia $\nu_{e w} = \bar{v_e}/R_0$ donde $\bar{v_e}$ es la velocidad media aritmética de los electrones. En cada choque existe una probabilidad de que ionice alguna molécula de la pared, extrayendo un electrón que contribuye a la avalancha, y una probabilidad de que se quede adherido a alguna de ellas, en decrimento de la avalancha. El primer caso ocurre con frecuencia $\nu_i = \dot{\mathcal{E}_e}/\Delta$ donde $\dot{\mathcal{E}_e}$ es la velocidad media de aumento de la energía cinética del electrón ${\mathcal{E}_e}$ debido al movimiento caótico. Mientras que $\Delta$ es la energía de ionización de una molécula de la pared. Esta puede ser mucho menor que la energía de ionización de una molécula individual del líquido y se estima en $\sim 6$ eV. \textcolor{red}{Bunkin referencia a Grand 1979. El paper ya lo descargué, debería leerlo}. El segundo caso ocurre con frecuencia $\nu_{loss} = (\bar{v}_e/R_0) w_{a,e}$ donde $w_{a,e}$ es la probabilidad de adhesión a la pared. De este modo, el número de electrones en el tiempo se encuentra determinado por la ecuación
\begin{equation}
  \frac{dN_e}{dt} = \nu_i N_e - \nu_{loss} N_e.
  \label{eq:dNedt_1}
\end{equation}


\ptitle{Determinación de la frecuencia de ionización}


A continuación se determinará la relación entre $\nu_i$ y las propiedades del láser. Asumiendo que el efecto del campo eléctrico es mayor al magnético, la velocidad $\dot{\mathcal{E}_e}$ corresponde a la potencia promedio $\langle P \rangle$ entregada por el haz de luz a un electrón. Además, considerando que la amplitud de oscilación del electrón es mucho menor a la longitud de onda del haz, el campo eléctrico $E(t)$ se puede considerar uniforme
\[E(t) = E_0 \sin{(w t)},\]
con $w$ frecuencia del haz y $E_0$ amplitud. En base a esto, la posición del electrón está dada clásicamente por la ley de Newton
\[m \dot{v} = -e E(t) -m \nu_{e w} v\]
donde $m$ es la masa del electrón, $v$ su velocidad, $e$ su carga y el término de fricción $m \nu_{e w} v$ representa las colisiones con la pared. Resolviendo la ecuación diferencial para tiempos largos $t \nu_{e w} >> 1$ se obtiene la velocidad
\[v(t) = \frac{e E_0}{m(\nu_{e w}^2 + w^2)}[w \cos{(t w)} - \nu_{e w} \sin{(t w)}]. \]
En base a esta se puede calcular la potencia promedio entregada al electrón 
\[\langle P \rangle = \langle - e E(t) v(t) \rangle = \nu_{e w}\frac{e^2 E_0^2}{2 m w^2} = \nu_{e w} \frac{I}{c n_{e \, cr}}\]
donde $n_{e \, cr} = mw^2/4 \pi e^2$ es la densidad de electrones crítica e $I = \frac{c}{8 \pi} E_0^2$ es la intensidad del haz, con $c$ velocidad de la luz. De este modo, la frecuencia de ionización se puede expresar como $\nu_i = \frac{\nu_{e w}}{\Delta} \frac{I}{c n_{e \, cr}}$

\ptitle{Determinación del nro de electrones en el tiempo y planteo de la condición límite}

Una vez calculada la frecuencia de ionización, se está en condiciones de determinar la evolución de $N_e(t)$. Reemplazando la expresión anterior en \ref{eq:dNedt_1}, se obtiene
\begin{equation}
  \frac{dN_e}{dt} = \nu_{e w} w_{a,e} \left ( \frac{I}{I_0} - 1 \right ) N_e = \beta N_e,
  \label{eq:dNedt_2}
\end{equation}
donde $I_0 = c n_{e \, cr} \Delta w_{a,e}$ es la intensidad mínima del haz necesaria para que $\beta>0$ y, por lo tanto, se produzca la avalancha de electrones. \textcolor{red}{Vale la pena hacer la cuenta con el láser del laboratorio?}. En caso de producirse, la expresión anterior indica un crecimiento exponencial $N_e(t) = N_e(0) e^{\beta t}$. Sin embargo, esto no puede ocurrir. El análisis anterior vale cuando la densidad media de electrones $\bar{n}_e$ es pequeña y por lo tanto se pueden despreciar las colisiones entre electrones, es decir, el camino libre medio de colisión entre electrones $l_{ee}$ es más grande que el tamaño de la burbuja $R_0$ que se asume que no varía. Cuando esto deja de valer, las colisiones de los electrones con la pared comienzan a disminuir, como así también $\nu_{e w}$ y, por lo tanto, el factor $\beta$ en \ref{eq:dNedt_2}. En la competencia de ambos procesos se llegaría a un equilibrio en el que la cantidad de electrones no aumenta. Para determinar tal número máximo de electrones $N_{e, max}$ es necesario conocer la densidad de electrones dentro del bubston.

\ptitle{Cálculo de la densidad media máxima}

A continuación se calculará la distribución de la densidad de electrones $n_e(r)$ dentro de un bubston. Asumiendo la validez de la distribución de Boltzmann \textcolor{red}{y qué otras hipótesis??. Este problema se conoce como el problema de Debye. Esto se discute en $https://en.wikipedia.org/wiki/Debye_length$} y aplicando la ecuación de Poisson, esta densidad es solución del sistema de ecuaciones diferenciales
\[
\left\{\begin{matrix}
  n_e(r) = n_e(0)e^{e \phi / T_e}, 0 \leq r \leq R_0 \\
  \triangledown^2 \phi = 4 \pi e n_e(r)
\end{matrix}\right.
\]
donde $\phi(r)$ es el potencial producido por la densidad de carga y $T_e$ es la temperatura del gas de electrones. Realizando la aproximación $e^{e \phi / T_e} \approx 1 + e \phi/T_e$
se logra despejar la ecuación diferencial para el potencial
\[\triangledown^2 \phi - \phi/a_e^2 = 4 \pi e n_e(0) \]
donde $a_e = \sqrt{T_e/4\pi e^2 n_e(0)}$ es el \textcolor{red}{electron Debye radius. Es lo mismo que el radio de Debye en un plasma?}. Imponiendo la condición de borde $\phi(0) = 0$ se obtiene la solución
\begin{equation}
  \phi(r) = \frac{T_e}{e} \left ( \frac{\sinh{r/a_e}}{r/a_e} - 1   \right ),
  \label{eq:cap1_phi}
\end{equation}
y la distribución de electrones
\begin{equation}
  n_e(r) = n_e(0) \frac{\sinh{r/a_e}}{r/a_e}
  \label{eq:cap1_ne}
\end{equation}


\ptitle{Cálculo de las ctes de la expresión de la densidad máxima y de $N_e$ y }

\begin{itemize}
  \item A partir de la definición de $a_e$ se obtiene una expresión para $n_e(0)$
  \item Solo falta calcular $a_e$
  \item Esto se puede hacer a partir de la relación
  \item \[N_e = 4 \pi \int_0^{R_0} r^2 n_e(r) dr \]
  \item 
\end{itemize}


\begin{itemize}
  \item \textcolor{blue}{Gráfico de la densidad de electrones. Se pueden poner letras como ticklabels en matplotlib $https://www.tutorialspoint.com/matplotlib/matplotlib_setting_ticks_and_tick_labels.htm$}
  \item 
  \item Relación entre $n_e(R_0)$ y $\bar{n_e}$.
\end{itemize}

\ptitle{Cálculo de $\bar{n}_{e \, max}$ y $N_{e \, max}$}
\begin{itemize}
  \item La densidad máxima se alcanza en la condición límite $l_{ee} = a_e$, obteniendo el valor máximo de la densidad y, por lo tanto, de la cantidad de electrones.
  
  \item Explicar en algún lado la la formación de la capa superficial y su justificación a partir del camino libre medio
  
\end{itemize}


\ptitle{Cálculo del tiempo al que se alcanza la densidad media máxima $t_0$}
\begin{itemize}
  \item \textcolor{red}{Basta decir que son tiempos cortos del orden de $6*10^{-11}$?? O tengo que hacer la cuenta? La cuenta es específica para esto y es muy larga}
\end{itemize}

\subsection{Coalescencia de los bubston y formación del núcleo}

\begin{itemize}
  \item Hasta ahora se vio que la cantidad de electrones dentro de la burbuja aumenta y al mismo ritmo se forma una capa superficial de cargas positivas debido a las moléculas ionizadas que rodean la burbuja.
  \item Cabe preguntarse qué efecto tiene la distribución de carga sobre la dinámica de la burbuja.
  \item Su efecto es el de producir una presión eléctrica sobre la pared de la burbuja, provocando su expansión. Tal presión $p_e$ se encuentra dada por
  \item \[p_e = $p_{gas}$ + $p_{coul}$\]
  \item donde $p_{gas} = (2/3)n_e(R)T_e(R)$ es la presión producida por el gas de electrones \textcolor{red}{por qué ese valor?} y $p_{coul} = E^2(R)/8 \pi$ es la presión debido a la repulsión coulombiana entre las cargas positivas y negativas. En esta última,
  \[E(R) = -d\phi/dr = \dots \] 
  es el campo eléctrico dentro de la burbuja.
  \item \textcolor{red}{Tengo que agregar bien cómo se hace la cuenta? No recuerdo bien el argumento de Fabián}
  \item De este modo,
  \item \textcolor{blue}{Expresión para la presión eléctrica final sin introducir el x}
  \item \textcolor{blue}{Gráfico de la presión para ver su dependencia con $1/R^4$}
\end{itemize}

Presentación de la ecuación diferencial de evolución de $R(t)$
\begin{itemize}
  \item A continuación se verá que la presión eléctrica produce que el bubston se expanda. Se parte de la ecuación de evolución de $R(t)$
  \item Esta es una aproximación de la ecuación de momento \textcolor{red}{Referencial el libro "The Acustic Bubble" de T.G.Leighton}
  \begin{equation}
    R \dot{u} + \frac{3}{2}u^2 = \frac{p_e}{\rho}
    \label{eq:bubston_ec_dif_R}
  \end{equation}
    \item donde $u = \dot{R} = dR/dt$ y se consideró que $p_e >> p_0$ presión hidrostática.)
  \item Lo único que queda determinar para resolver la ecuación diferencial es la dependencia $n_e(R)$ y $T_e(R)$.
\end{itemize}


Definición del régimen de autoconsistencia
\begin{itemize}
  \item Hay que suponer que $\tau >> t_0$
  \item El proceso de expansión se considera como cuasiestático. Esto implica que la temperatura del gas depende del radio $T = T(R)$ y que la distribución de electrones cambia con $R$ y $T(R)$ solamente.
  \item Bajo estas condiciones, el número máximo de electrones $N_{e \, max}(R)$ \textcolor{red}{verificar fórmula. cambié la def de $a_e$} debería cambiar para cada valor de $R = R(t)$.
  \item Sin embargo, físicamente se espera que no cambie en el tiempo debido a los argumentos presentados anteriormente y a la consideración del proceso como cuasiestático.
  \item En base a esto, se define el proceso de expansión como un "régimen de autoconsistencia" en el que la cantidad de electrones no cambia con el tiempo.
  \item Para que esto ocurra, se debe cumplir que
  \item \[T_e(R) = T_e(R_0)\frac{R_0}{R} \]
  \item donde $T_e(R_0) = \Delta/3$
\end{itemize}

Dinámica de la burbuja para $R>R_0$
\begin{itemize}
  \item Dada la funcionalidad conocida de $p_e(R)$, lo único que queda es resolver la ecuación diferencial \ref{eq:bubston_ec_dif_R}.
  \item Esto se hizo numéricamente mediante el método Runge-Kutta-4.
  \item \textcolor{red}{Tengo que explicar en detalle cómo resolví numéricamente el problema?}
  \item \textcolor{blue}{Figura de R(t) marcando el límite en el que hay coalescencia}
  \item Esta evolución no vale indefinidamente, sino hasta que se produce la coalescencia de los bubstons del mismo cluster. Esto ocurre cuando $R(t) = R_1=l/2$ donde $l$ es la distancia entre los centros de los bubstons. Esta longitud se puede estimar como $l \approx 2a_i$
  \item \textcolor{red}{Revisar referencia y definir $a_i$}
  \item Numéricamente se obtuvo que el tiempo en el que ocurre este proceso $t_{coal}$ es del orden de \textcolor{red}{valor?}
\end{itemize}

\section{Evolución de la burbuja}

Armo el cuento desde la coalesciancia hasta llegar a los resultados importantes de radio máximo y el tiempo al que se da

\begin{itemize}
  \item Luego del tiempo de coalescencia se asume que todos los bubstons se fusionan para formar una burbuja grande, a partir de ahora denominada "núcleo".
  \item Condciones iniciales de la burbuja
  \item Calculo los distintos parámetros en función de las fórmulas presentadas
  \item 
  \item Se busca calcular la presión eléctrica en $R = R_{cl}$ a tiempo inicial. La presión eléctrica está dada por la ecuación (17) del Bunkin
  \[p_{e} = \frac{2}{3}\bar{n_e} T_e \frac{x}{3} \left [1 + \pi \left ( \frac{x-1}{x} \right )^2 \frac{e^2R_{cl}^2\bar{n_e}}{T_e}   \right ] \]
  donde $\bar{n_e} = 3.4 *10^{16} \, \mathrm{cm}^{-3}$ es la densidad media de electrones en el cluster, $T_e \approx 0.1 \, \mathrm{eV}$ es la temperatura de los electrones, $x \approx 17.9$ es un parámetro, $e$ es la carga eléctrica y $R_{cl}$ es el radio del cluster que a $t=0$ es $R_{cl} \approx 10^{-2} \, \mathrm{cm}$.  

  \item A partir de ahora puede dejar de valer el "régimen de autoconsistencia", por lo que la dependencia de $T(R)$ presentada en \textcolor{red}{referencia?} podría dejar de valer. En este sentido, se asume que los electrones del núcleo se encuentran a la misma temperatura correspondiente al instante anterior a la coalescencia, es decir, $T(R_1) \approx 0.1 eV$.



  \item Debido a que no se conoce el proceso por el cual los bubstons forman el núcleo, tampoco se puede asumir que la distribución de electrones $n_e(r)$ mantenga la funcionalidad de la expresión \textcolor{red}{referencia}. Consecuentemente, la expresión para la presión eléctrica también deja de valer. Aún más, la ecuación de evolución \ref{eq:bubston_ec_dif_R} deja de valer debido a que durante la dinámica del clúster podría ocurrir que la presión eléctrica sea comparable con la hidrostática externa. 

  \item Sin embargo, sí sepuede asumir $\dots$ \textcolor{red}{para continuar tengo que repasar las condiciones de la fórmula de Willis}
  \item \textcolor{red}{condiciones para la fórmula de Willis. Buscar bien y explicar por qué se plantea la igualdad de energías a tiempo inicial. No me queda claro por qué deja de valer la evolución que hicimos con la presión de electrones.}

  \item 
  \begin{equation}
    R \dot{u} + \frac{3}{2}u^2 = \frac{p_0}{\rho}
    \label{eq:bubston_ec_dif_R}
  \end{equation}


  \item La velocidad inicial está dada por 
  \item \textcolor{blue}{fórmula de la velocidad inicial en función de la energía}
  \item 
  \item Mientras que las condiciones iniciales son 
  \item Se obtuvo el radio máximo $\mathrm{R_{max}} = $ \textcolor{red}{valor?} cm a tiempo $\mathrm{t_{max}} = $ \textcolor{red}{valor?} s.
\end{itemize}






























\section*{Capítulo II: Evolución desde el radio máximo}

\subsection{Resumen del capítulo}



\begin{itemize}
  \item Resumen de los fenómenos que participan en esta parte de la evolución (están en la tesis de Gabriela).
  \item El objetivo es calcular la evolución del radio $R(t)$ de la burbuja desde su valor máximo.
  \item Hay que resolver la ecuación de movimiento \textcolor{blue}{Presentar ecuación, de dónde se obtuvo, qué es cada factor y de dónde se sacan los valores, son los mismos para deuterio e hidrógeno?}
  \item La variación de masa $\dot{m}$ está dada por 3 fenómenos: reacciones químicas, difusión y condensación y evaporación. \textcolor{red}{Por qué en la tesis de gabriela solo hay variación de mp por condensación y evaporación}
  \item Comentar que en este trabajo se logró implementar el código de reacciones y de condensación/evaporación.
  \item Por otro lado, para calcular la variación de $p_B$ es necesario calcular $dT/dt$ para lo cual se necesita una expresión. Aquí es donde se introuce la conservación de la energía.
  \item Comentar qué modelo se considera para la presión de los gases dentro de la burbuja y qué modelo para la energía
  \item Comentar que se trata de un problema del tipo stiff y deberá ser trabajado con cuidado. No definir problemas stiff, eso conviene hacerlo en otra sección.
\end{itemize}

\subsection{Fenómenos físicos}




\subsubsection{Condensación y Evaporación}

\begin{itemize}
  \item Tengo que repasar la tesis de Gabriela y el paper de Yasui para comenzar a escribir esto
  \item Sí se podría hacer mención a las ecuaciones para calcular la conducción de calor.
\end{itemize}



\subsubsection{Reacciones químicas}

Gabriela obtuvo el modelo de reacciones qcas del paper de Yasui 1997. \textcolor{red}{Podría verificar si no hay un mejor modelo}.
Yasui escribió un libro sobre burbujas. Quicás sea útil leerlo. Mi principal foco estaría en el capítulo 2. Se desarrollan no solo las reacciones químicas sino también los demás fenómenos.

Toegel Phase diagrams for sonoluminescing bubbles: A comparison between experiment and theory. En este se desarrolla el modelo hidrondinámico de la burbuja, junto tmb a modelos de transferencia de calor. En cuanto a las reacciones químicas en particular, el modelo no es exactamente el de la ecuación de Arrhenius, sino de la "ley de Arrhenius modificada". En teoría esto se explica en mayor detalle en el libro "Gas-Phase Combustion Chemistry" de W. C. Gardiner pero no me puse a buscarlo



\begin{itemize}
  \item Releer la parte de reacciones químicas de la tesis de Gabriela y el paper de Yasui al que se hace referencia. Esta sección intenta resumir eso.
  \item \textcolor{red}{DUDA: en el código de Gabriela, n es nro de partículas o concentración? En la tesis es concentración}
  \item \textcolor{red}{Vale la pena agregar acá la contribución de las reacciones químicas a la conservación de la energía?}
\end{itemize}


\begin{itemize}
  \item Las reacciones químicas dentro de la burbuja son de no equilibrio \textcolor{red}{Referencia al libro de Yasui} y es necesario emplear un modelo de cinética química
  \item Ocurren cuando las temperaturas en el interior de la burbuja son elevadas
  \item En este trabajo se consideró sólo la presencia inicial de moléculas de $\mathrm{H_2O}$, $\mathrm{O_2}$ y $\mathrm{H_2}$, lo cual da lugar a \textcolor{red}{8} reacciones+
  \item El cambio total 

\end{itemize}


\subsection{Método numérico}

\begin{itemize}
  \item Mencionar que es un problema del tipo stiff. Explicar brevente de qué se trata esto
  \item Historia de cómo el código pasó de ejecutar en días (el mío) o minutos (el de Gabriela) a segundos (el mío):
  \item Comentar que ya se contaba con un código en C++ programado en Borland por Gabriela (y en qué año), pero que resolvía el sistema de ecuaciones diferenciales usando Runge-Kutta Ferhber \textcolor{red}{corroborar cómo se escribía}. Comentar qué complicaciones tenía y que se progrmó toro código en C++
  \item Se usó el cluster.
  \item Se optó por python. Se usó la librería scipy. Nuevamente el código era muy lento con RK45 pero con el método de Radau todo mejoró. \textcolor{red}{Explico esto?}
\end{itemize}















































\section*{Conclusiones}
\begin{enumerate}
  \item Se calculó numéricamente el radio máximo de la burbuja, obteniendo un valor 10 veces mayor a los resultados experimentales (y teóricos del Bunkin)
  \item Se calculó numéricamente el tiempo al que la burbuja alcanza el radio máximo, obteniendo un valor 10 veces mayor a los resultados experimentales (y teóricos del Bunkin)
  \item Ídem para la energía, salvo por un factor 600
  \item Se encontró que el problema es de naturaleza stiff, lo cual permitió elegir un mejor método numérico que el que se venía usando. Esto permitió disminuir enormemente el tiempo de cómputo
  \item Se logró calcular la evolución de un único bubston, obteniendo los mismos resultados que el paper de Bunkin
  \item Se logró implementar un módulo de reacciones, de conducción del calor y de condensación y evaporación.
  \item Se pudo usar el cluster de MECOM para agilizar las cuentas del código en C++, aunque luego se decidió trabajar en python
  \item Se lograron reproducir las  cuentas más importantes del Bunkin con Mathematica



\end{enumerate}




\bibliography{Chehade_tesis.bib}

\end{document}






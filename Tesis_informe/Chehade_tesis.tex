\documentclass[aps,prb,twocolumn,superscriptaddress,floatfix,longbibliography,10pt]{revtex4-2}

\usepackage[utf8]{inputenc}
\usepackage[spanish]{babel}
\usepackage{graphicx}
\usepackage{amsmath}
\usepackage{subcaption}
\usepackage{wrapfig} 
\usepackage[export]{adjustbox}

\usepackage{amsmath,amssymb} % math symbols
\usepackage{bm} % bold math font
\usepackage{graphicx} % for figures
\usepackage{comment} % allows block comments
\usepackage{textcomp} % This package is just to give the text quote '
%\usepackage{ulem} % allows strikeout text, e.g. \sout{text}

\usepackage[spanish]{babel}
% By dafault, spanish changes to a comma as decimal separator; to change to a dot, you can use \decimalpoint:
\decimalpoint

\usepackage{enumitem}
\setlist{noitemsep,leftmargin=*,topsep=0pt,parsep=0pt}

\usepackage{xcolor} % \textcolor{red}{text} will be red for notes
\definecolor{lightgray}{gray}{0.6}
\definecolor{medgray}{gray}{0.4}

%Para las tablas
\usepackage{multirow}

\usepackage{hyperref}
\hypersetup{
colorlinks=true,
urlcolor= blue,
citecolor=blue,
linkcolor= blue,
bookmarks=true,
bookmarksopen=false,
}

% Code to add paragraph numbers and titles
\newif\ifptitle
\newif\ifpnumber
\newcounter{para}
\newcommand\ptitle[1]{\par\refstepcounter{para}
{\ifpnumber{\noindent\textcolor{lightgray}{\textbf{\thepara}}\indent}\fi}
{\ifptitle{\textbf{[{#1}]}}\fi}}
\ptitletrue  % comment this line to hide paragraph titles
\pnumbertrue  % comment this line to hide paragraph numbers

% minimum font size for figures
\newcommand{\minfont}{6}

% Uncomment this line if you prefer your vectors to appear as bold letters.
% By default they will appear with arrows over them.
% \renewcommand{\vec}[1]{\bm{#1}}

%Cambiar Cuadros por Tablas y lista de...
%\renewcommand{\listtablename}{Índice de tablas}
\renewcommand{\tablename}{Tabla}
\renewcommand{\date}{Fecha}

%Para importar imágenes desde una carpeta:
\graphicspath{ {C:/Users/lupam/OneDrive/Escritorio/GitHub/Vapor_en_burbuja_inicial_python/Tesis_informe/Figures}}

\usepackage[bottom]{footmisc} %para que las notas al pie aparezcan en la misma página

\begin{comment}

%Comandos de interés:

* Para ordenar el documento:
\section{Introducción}
\section{\label{sec:Formatting}Formatting} %label para luego hacer referencia a esa sección

\ptitle{Start writing while you experiment} %pone nombre y título al documento dependiendo de si en el header están los comandos \ptitletrue y \pnumbertrue

* Ecuaciones:
\begin{equation}
a^2+b^2=c^2 \,.
\label{eqn:Pythagoras}
\end{equation}

* Conjunto de ecuaciones:
\begin{eqnarray}
\label{eqn:diagonal}
\nonumber d & = & \sqrt{a^2 + b^2 + c^2} \\
& = & \sqrt{3^2+4^2+12^2} = 13
\end{eqnarray}

* Para hacer items / enumerar:
\begin{enumerate}
  \item
\end{enumerate}

\begin{itemize}
  \item
\end{itemize}

* Figuras:
\begin{figure}[h]
    \includegraphics[clip=true,width=\columnwidth]{pixel-compare}
    \caption{}
     \label{fig:pixels}
\end{figure}

* Conjunto de figuras:
(no recuerdo)


* Para hacer referencias a fórmulas, tablas, secciones, ... dentro del documento:
\ref{tab:spacing}

* Para citar
Elementos de .bib
\cite{WhitesidesAdvMat2004}
url
\url{http://www.mendeley.com/}\\

* Agradecimientos:
\begin{acknowledgments}
We acknowledge advice from Jessie Zhang and Harry Pirie to produce Fig.\ \ref{fig:pixels}.
\end{acknowledgments}

* Apéndice:
\appendix
\section{\label{app:Mendeley}Mendeley}

* Bibliografía:
\bibliography{Hoffman-example-paper}

\end{comment}


\begin{document}

% Allows to rewrite the same title in the supplement
\newcommand{\mytitle}{\textcolor{red}{Cálculos Computacionales en Plasmas Producidos por Cavitación Láser y colapso de burbujas - trabajo realizado hasta 27 de Octubre de 2022}}

\title{\mytitle}

\author{Pablo Chehade \\
    \small \textit{pablo.chehade@ib.edu.ar} \\
    \small \textit{Instituto Balseiro, CNEA-UNCuyo, Bariloche, Argentina, 2022} \\}


\begin{abstract}
  \begin{itemize}
    \item Se estudió numéricamente la expansión de una burbuja de cavitación láser. \textcolor{red}{y la compresión}
    \item La evolución temporal se divide en dos partes, cada una de ellas gobernada por fenómenos físicos distintos.
    \item La primera consta desde la creación de la burbuja hasta la expansión rápida al radio máximo. El fenómeno que creemos ocurre es la interacción del láser con los electrones de burbujas microscópicas (bubston) para dar lugar a una avalancha de electrones impulsora de la expansión (fenómeno electromagnético y fluidodinámico). El fenómeno se debe principalmente a la acción de los electrones y las moléculas ionizadas que rodean exteriormente la burbuja.
    \item La segunda, desde el radio máximo hasta su implosión al radio mínimo. En este momento el efecto de los electrones se diluye y comienzan a preponderar efectos de conducción del calor, cambios en la masa contenida dentro de la esfera debido a reacciones químicas, difusión y condensación/evaporación, entre otros. El fenómeno se debe principalmente a la acción de los iones y moléculas dentro y fuera de la burbuja (fenómeno fluidodinámico, termodinámico y químico).
    \item Se exploraron además distintos métodos numéricos y técnicas de resolución computacionales para resolver este problema complejo que inherentemente es del tipo stiff debido a las diferencias de órdenes de magnitud entre las constantes de tiempo involucradas.
  \end{itemize}


\end{abstract}

\maketitle

\section*{Índice}
\textcolor{red}{Buscar en internet cómo hacer un índice en Latex}

\section*{Introducción}


















\section*{Capítulo I: Evolución hasta el radio máximo}

\ptitle{Resumen del capítulo}
\begin{itemize}
  \item En este capítulo se discute el proceso a partir del cual un láser incidente en un medio líquido es capaz de producir una burbuja de cavitación
  \item El proceso se desarrolla en \textcolor{red}{Referencia al Bunkin}
  \item Se parte de un medio líquido, en nuestro caso agua deuterada, en el cual se asume existen burbujas de gas (bubstons) agrupadas en clusters en la zona focal del haz de luz. 
  \item Se asume que dentro de cada bubston hay inicialmente un electrón libre
  \item Bajo estas condiciones, se demuestra en el paper que bajo determinadas condiciones de la burbuja y de la intensidad del haz, se desarrolla una avalancha de electrones
  \item Esta produce una presión eléctrica que expande el bubston hasta que los bubstons del cluster se fusionan entre sí y forman una gran burbuja o "núcleo". 
  \item Este efecto se denomina SOC
  \item Explicar brevemente qué ocurre luego
  \item Las cuentas se hacen en el sistema CGS

\end{itemize}




Armo el cuento hasta la coalescencia


\subsection{Situación física inicial}
\begin{itemize}
  \item Resumen: en esta sección se explica la situación física inicial antes de la incidencia del láser.
  \item Se parte de un medio líquido en el que se encuentran clusters de burbujas estables (bubstons) de radio $R_0$.
  \item Explicar cómo calcular R0 y dar el valor aproximada
  \item La estabilidad de los bubstons se basa en la condición $R_0 << l_{em}$. Explicar
  \item Se considera que dentro de la burbuja hay al menos un electrón libre
  \item Definir R0 en algún lado
\end{itemize}

\subsection{Avalancha de electrones dentro de los bubston}

\ptitle{Resumen}
\begin{itemize}
  \item En esta sección se explicará el proceso a través del cual aumenta la cantidad de electrones $N_e$  dentro de cada bubston.
  \item También se verá que para que ocurra la avalancha es necesario que el haz supere una determinada intensidad \textcolor{red}{dada por qué parámetros? del líquido?}
  \item Además, se verá que durante la avalancha no aumenta la cantidad de electrones indefinidamente, sino que se llega a un valor máximo en un tiempo relativamente rápido
\end{itemize}

\ptitle{La avalancha comienza con un electrón dentro de la burbuja inmerso en el campo electromagnético producido por el láser}

\begin{itemize}
  \item La situación física inicial corresponde a un electrón dentro de un bubston inmerso en un campo electromagnético oscilatorio.
  \item Debido al campo, el electrón se va a mover y va a chocar con las paredes de la burbuja con una frecuencia $\nu_{e w} = \bar{\nu_e}/R_0$ donde $\bar{\nu_e}$ es la velocidad media aritmética de los electrones.
  \item En cada choque existe una probabilidad de que ionice alguna molécula de la pared, contribuyendo a la avalancha, y una probabilidad de que se quede adherido a alguna  de ellas, en decrimento de la avalancha.
  \item El primer caso ocurre con frecuencia $\nu_i = \dot{\mathcal{E}_e}/\Delta$ donde $\dot{\mathcal{E}_e}$ es la velocidad media de aumento de la energía cinética del electrón ${\mathcal{E}_e}$ debido al movimiento caótico. Mientras que $\Delta$ es la energía de ionización de una molécula de la pared. Esta puede ser mucho menor que la energía de ionización de una molécula individual del líquido y se estima en $\sim 6$ eV. \textcolor{red}{Leer la referencia 4 en el Bunkin}. 
  \item El segundo caso ocurre con frecuencia $(\bar{\nu}_e/R_0) w_{a,e}$ donde $w_{a,e}$ es la probabilidad de adhesión a la pared. 
\end{itemize}

\ptitle{Determinación de la frecuencia de ionización}
\begin{itemize}
  \item Asumiendo que el efecto del campo eléctrico es mayor al magnético, la velocidad $\dot{\mathcal{E}_e}$ corresponde a la potencia entregada por el haz a un electrón.
  \item Además, considerando que la amplitud de oscilación del electrón es mucho menor a la longitud de onda del haz, el campo eléctrico se puede considerar uniforme.
  \item Entonces, el campo eléctrico se puede expresar como $E(t) = E_0 \sin{(w t)}$, con $w$ frecuencia del haz y $E_0$ amplitud.
  \item Para calcular la potencia es necesario analizar el movimiento del electrón
  \item La posición del electrón está dada clásicamente por la ley de newton
  \item $m \dot{v} = -e E(t) -m \nu_{e w} v $
  \item donde $m$ es la masa del electrón, $v$ su velocidad y $e$ su carga. En esta ecuación las colisiones con la pared se tuvieron en cuenta a partir del término de fricción $m \nu_{e w} v$. 
  \item Para tiempos largos (\textcolor{red}{definir esto}) se puede resolver la ecuación diferencial obteniendo la velocidad
  \item \textcolor{blue}{expresión de la velocidad de la cuenta de Gustavo}
  \item En base a esta velocidad se puede calcular la potencia entregada por el campo al electrón
\end{itemize}

\ptitle{Determinación de la frecuencia de pérdida de electrones}
\begin{itemize}
  \item test
\end{itemize}

\ptitle{Determinación del nro de electrones en función del tiempo}
\begin{itemize}
  \item Si la frecuencia de ionización es mayor a la de pérdida de electrones, se desarrolla una avalancha.
  \item De este modo, a tiempos cortos (\textcolor{red}{para los que no creció demasiado el radio y se puede asegurar que solo ocurren los dos fenómenos anteriores?})
  \item \textcolor{blue}{Expresión (2) del Bunkin}
  \item donde \textcolor{blue}{derivación de $\theta$}.
\end{itemize}

\ptitle{Condición mínima para la intensidad del haz}
\begin{itemize}
  \item Esto da además una condición sobre la intensidad del haz. \textcolor{red}{Hacer una cuenta con el láser del laboratorio}
\end{itemize}

\ptitle{Aumento indefinido de electrones?}

\begin{itemize}
    \item Esto demuestra que aumentan los electrones y se forma dentro de la burbuja un gas de electrones.
    \item Además, en la superficie se forma una esfera de carga uniforme positiva debido a los iones
    \item La anterior evolución vale solo cuando la densidad media de electrones es pequeña y por lo tanto el camino libre medio de colisión entre electrones es más grande que el tamaño de la burbuja.
    \item Cuando esto deja de valer, las colisiones de los electrones con la pared deben disminuir y por lo tanto debería disminuir la cantidad de electrones.
    \item En la competencia de ambos procesos debería llegarse a un equilibrio en el que la cantidad de electrones no aumenta.
    \item Este justificativo físico es necesario para continuar el análisis
    \item El tiempo en el que ocurre esto y el valor de la densidad media máxima será determinado próximamente
\end{itemize}

\ptitle{Cálculo de la densidad media máxima}

\begin{itemize}
  \item Debido a la interacción coulombiana la densidad de electrones $n_e(r)$ debe ser no uniforme
  \item Determinación instantánea de la densidad de electrones y el potencial eléctrico
  \item Gráficos de ambas funciones
\end{itemize}


\ptitle{Cálculo del tiempo al que se alcanza la densidad media máxima $t_0$}



\subsection{Coalescencia de los bubston y formación del núcleo}

\begin{itemize}
  \item Se busca calcular la evolución del radio. Para esto se cuenta con la ecuación diferencial aproximada
  \item donde en la término de presión hay que considerar tanto la hidrostática como la de los electrones
  \item Se parte de un gas de electrones dentro de la burbuja. 
  \item Explicar la formación de la capa superficial y su justificación a partir del camino libre medio
  \item Determinación del campo eléctrico
  \item Determinación de la presión eléctrica en la superficie de la burbuja ($p_{gas}$ + $p_{coulomb}$)
  \item Lo único que queda determinar es la temperatura. Para
\end{itemize}


\section{Evolución de la burbuja}

Armo el cuento desde la coalesciancia hasta llegar a los resultados importantes de radio máximo y el tiempo al que se da

\begin{itemize}
  \item Luego del tiempo de coalescencia se asume que todos los bubstons se fusionan para formar una burbuja grande, a partir de ahora denominada "núcleo".
  \item Condciones iniciales de la burbuja
  \item Se busca calcular la presión eléctrica en $R = R_{cl}$ a tiempo inicial. La presión eléctrica está dada por la ecuación (17) del Bunkin
  \[p_{e} = \frac{2}{3}\bar{n_e} T_e \frac{x}{3} \left [1 + \pi \left ( \frac{x-1}{x} \right )^2 \frac{e^2R_{cl}^2\bar{n_e}}{T_e}   \right ] \]
  donde $\bar{n_e} = 3.4 *10^{16} \, \mathrm{cm}^{-3}$ es la densidad media de electrones en el cluster, $T_e \approx 0.1 \, \mathrm{eV}$ es la temperatura de los electrones, $x \approx 17.9$ es un parámetro, $e$ es la carga eléctrica y $R_{cl}$ es el radio del cluster que a $t=0$ es $R_{cl} \approx 10^{-2} \, \mathrm{cm}$.  
  \item La velocidad inicial está dada por 
  \item \textcolor{red}{condiciones para la fórmula de Willis. Buscar bien y explicar por qué se plantea la igualdad de energías a tiempo inicial. No me queda claro por qué deja de valer la evolución que hicimos con la presión de electrones.}
  \item Ecuación diferencial de evolución del radio de la burbuja
  \item Esta es una aproximación de la ecuación de momento \textcolor{red}{Referencial el libro "The Acustic Bubble" de T.G.Leighton}
  \item teniendo en cuenta como presión a la hidrostática (el interior de la burbuja se considera vacío).
  \item Mientras que las condiciones iniciales son 
  \item Se obtuvo el radio máximo $\mathrm{R_{max}} = $ \textcolor{red}{valor?} cm a tiempo $\mathrm{t_{max}} = $ \textcolor{red}{valor?} s.
\end{itemize}






























\section*{Capítulo II: Evolución desde el radio máximo}

\subsection{Resumen del capítulo}



\begin{itemize}
  \item Resumen de los fenómenos que participan en esta parte de la evolución (están en la tesis de Gabriela).
  \item El objetivo es calcular la evolución del radio $R(t)$ de la burbuja desde su valor máximo.
  \item Hay que resolver la ecuación de movimiento \textcolor{blue}{Presentar ecuación, de dónde se obtuvo, qué es cada factor y de dónde se sacan los valores, son los mismos para deuterio e hidrógeno?}
  \item La variación de masa $\dot{m}$ está dada por 3 fenómenos: reacciones químicas, difusión y condensación y evaporación. \textcolor{red}{Por qué en la tesis de gabriela solo hay variación de mp por condensación y evaporación}
  \item Comentar que en este trabajo se logró implementar el código de reacciones y de condensación/evaporación.
  \item Por otro lado, para calcular la variación de $p_B$ es necesario calcular $dT/dt$ para lo cual se necesita una expresión. Aquí es donde se introuce la conservación de la energía.
  \item Comentar qué modelo se considera para la presión de los gases dentro de la burbuja y qué modelo para la energía
  \item Comentar que se trata de un problema del tipo stiff y deberá ser trabajado con cuidado. No definir problemas stiff, eso conviene hacerlo en otra sección.
\end{itemize}

\subsection{Fenómenos físicos}

\subsubsection{Reacciones químicas}

Gabriela obtuvo el modelo de reacciones qcas del paper de Yasui 1997. \textcolor{red}{Podría verificar si no hay un mejor modelo}.
Yasui escribió un libro sobre burbujas. Quicás sea útil leerlo. Mi principal foco estaría en el capítulo 2. Se desarrollan no solo las reacciones químicas sino también los demás fenómenos.

Toegel Phase diagrams for sonoluminescing bubbles: A comparison between experiment and theory. En este se desarrolla el modelo hidrondinámico de la burbuja, junto tmb a modelos de transferencia de calor. En cuanto a las reacciones químicas en particular, el modelo no es exactamente el de la ecuación de Arrhenius, sino de la "ley de Arrhenius modificada". En teoría esto se explica en mayor detalle en el libro "Gas-Phase Combustion Chemistry" de W. C. Gardiner pero no me puse a buscarlo



\begin{itemize}
  \item Releer la parte de reacciones químicas de la tesis de Gabriela y el paper de Yasui al que se hace referencia. Esta sección intenta resumir eso.
  \item \textcolor{red}{DUDA: en el código de Gabriela, n es nro de partículas o concentración? En la tesis es concentración}
  \item \textcolor{red}{Vale la pena agregar acá la contribución de las reacciones químicas a la conservación de la energía?}
\end{itemize}


\begin{itemize}
  \item Las reacciones químicas dentro de la burbuja son de no equilibrio \textcolor{red}{Referencia al libro de Yasui} y es necesario emplear un modelo de cinética química
  \item Ocurren cuando las temperaturas en el interior de la burbuja son elevadas
  \item En este trabajo se consideró sólo la presencia inicial de moléculas de $\mathrm{H_2O}$, $\mathrm{O_2}$ y $\mathrm{H_2}$, lo cual da lugar a \textcolor{red}{8} reacciones+
  \item El cambio total 

\end{itemize}




\subsubsection{Condensación y Evaporación}

\begin{itemize}
  \item Tengo que repasar la tesis de Gabriela y el paper de Yasui para comenzar a escribir esto
  \item Sí se podría hacer mención a las ecuaciones para calcular la conducción de calor.
\end{itemize}

\subsection{Método numérico}

\begin{itemize}
  \item Mencionar que es un problema del tipo stiff. Explicar brevente de qué se trata esto
  \item Historia de cómo el código pasó de ejecutar en días (el mío) o minutos (el de Gabriela) a segundos (el mío):
  \item Comentar que ya se contaba con un código en C++ programado en Borland por Gabriela (y en qué año), pero que resolvía el sistema de ecuaciones diferenciales usando Runge-Kutta Ferhber \textcolor{red}{corroborar cómo se escribía}. Comentar qué complicaciones tenía y que se progrmó toro código en C++
  \item Se usó el cluster.
  \item Se optó por python. Se usó la librería scipy. Nuevamente el código era muy lento con RK45 pero con el método de Radau todo mejoró. \textcolor{red}{Explico esto?}
\end{itemize}















































\section*{Conclusiones}
\begin{enumerate}
  \item Se calculó numéricamente el radio máximo de la burbuja, obteniendo un valor 10 veces mayor a los resultados experimentales (y teóricos del Bunkin)
  \item Se calculó numéricamente el tiempo al que la burbuja alcanza el radio máximo, obteniendo un valor 10 veces mayor a los resultados experimentales (y teóricos del Bunkin)
  \item Ídem para la energía, salvo por un factor 600
  \item Se encontró que el problema es de naturaleza stiff, lo cual permitió elegir un mejor método numérico que el que se venía usando. Esto permitió disminuir enormemente el tiempo de cómputo
  \item Se logró calcular la evolución de un único bubston, obteniendo los mismos resultados que el paper de Bunkin
  \item Se logró implementar un módulo de reacciones, de conducción del calor y de condensación y evaporación.
  \item Se pudo usar el cluster de MECOM para agilizar las cuentas del código en C++, aunque luego se decidió trabajar en python
  \item Se lograron reproducir las  cuentas más importantes del Bunkin con Mathematica



\end{enumerate}




\bibliography{Chehade_tesis.bib}

\end{document}





